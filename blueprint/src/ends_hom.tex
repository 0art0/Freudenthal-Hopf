\chapter{Functoriality of Ends}

\section{Functoriality}

Consider two graphs $G$ and $G'$ with vertex sets $V$ and $V'$.

Given a graph homomorphism $\phi : G \to G'$, we would like to capture conditions on $\phi$ for required to obtain a map between the ends of the graphs.

The definition of \emph{ends} of a graph $G$ used here is as the sections of a functor from the poset of finite subsets of $V$ to the category of sets sending each finite subset $K$ to a connected component in its complement.

The rough strategy for inducing a map between ends is as follows:
\begin{enumerate}
    \item Consider a \emph{finite} set $L \subset V'$ in the target.
    \item Use $\phi$ to produce a corresponding \emph{finite} subset $K \subset V$.
    \item An end $e$ of $G$ assigns to $K$ a connected component $C$ in the complement of $K$.
    \item If $\phi$ is nice enough, the image of $C$ under $\phi$ is contained in a unique component $C'$ of $G'$.
    \item The end of $G'$ corresponding to $e$ is the one sending $L$ to $C'$.
\end{enumerate}

\section{Cofinite maps}

\begin{definition}
    \label{def:cofinite_map}
    A function $f : V \to W$ is said to be \emph{cofinite} whenever the pre-images of finite subsets of $W$ under $f$ are finite subsets of $V$.
\end{definition}
  
When a map $\phi$ between graphs is cofinite, the second step in the strategy above can be to just set $K := \phi^{-1}(L)$.

We will focus on the case of cofinite functions inducing a map between ends. It turns out, though it will not be proved here, that any function inducing a map between ends in this way is necessarily cofinite.

\section{Weak homomorphisms}

\begin{definition}
    \label{def:reachable}
    Given a graph $G$ on a vertex set $V$, two points $a$ and $b$ of $V$ are \emph{reachable} if there exists a path between them.
  \end{definition}
  
  \begin{definition}
    \label{def:weak_graph_hom}
    A \emph{weak homomorphism} between two graphs $G$ and $G'$ is a function $f : G \to G'$ that sends reachable pairs of points to reachable pairs of points.
  \end{definition}
  
  \begin{theorem}
    \label{thm:weak_hom_iff_conn_map}
    A map $f : G \to G'$ between graphs induces a map between their connected components iff and only if $f$ is a weak homomorphism.
  \end{theorem}
  
  \begin{proof}
  \texttt{sorry}
  \end{proof}
  
  In the fourth step of the strategy, it suffices if $\phi : (G \backslash f^{-1}(L)) \to (G' \backslash L)$ is a weak homomorphism for each finite $L \subset V'$.

\section{Reduction}

This section gives a convenient description of maps that induce maps between ends, in the case where the target is locally-finite.

\begin{theorem}
Suppose $\phi : G \to G'$ is a cofinite function between \textbf{locally-finite} graphs such that $\phi : (G \backslash f^{-1}(L)) \to (G' \backslash L)$ is a weak homomorphism for each finite $L \subset V'$. Then $\phi$ is a homomorphism from $G$ to $G'$.
\end{theorem}

\begin{proof}
To prove that $\phi$ is a graph homomorphism, we need to show that if two vertices $a, b \in G$ are adjacent, then $\phi(a), \phi(b) \in G'$ are adjacent. It suffices to show that there is a path between $\phi(a)$ and $\phi(b)$ that involves no additional vertices of $G'$.

Motivated by this, we define $L := (N(\phi(a)) \cup N(\phi(b))) \backslash \{\phi(a), \phi(b)\}$ ($N$ denotes the neighbourhood). When the graph $G'$ is locally finite, this set is a finite subset of $V'$. The construction ensures that any path between $\phi(a)$ and $\phi(b)$ in $G' \backslash L$ is necessarily an edge, as these vertices have no other neighbours.

In the graph $G \backslash \phi^{-1}(L)$, $a$ and $b$ remain adjacent if they were initially adjacent in $G$. By the assumption, $\phi : (G \backslash \phi^{-1}(L)) \to (G' \backslash L)$ is a weak homomorphism. We have that $a$ and $b$ are reachable in $G \backslash \phi^{-1}(L)$, as they are adjacent. Therefore there is a path between $\phi(a)$ and $\phi(b)$ in $G \backslash \phi^{-1}(L)$, by the weak homomorphism property. However, the construction of $L$ forces $a$ and $b$ to be adjacent vertices in $G'$.

This shows that $\phi$ maps adjacent vertices to adjacent vertices, or in other words, is a graph homomorphism.
\end{proof}




